\documentclass[a4paper,10pt]{article}
\usepackage[utf8]{inputenc}
\usepackage{tikz}

\newcommand{\icon}[1]{\tikz[baseline=-3pt] \node[inner sep=0pt,outer sep=0pt]{\includegraphics[height=1.1em]{images/#1}};}

%opening
\title{ADAMS - Scientific Workflow Management}
\author{Peter Reutemann}

\begin{document}

\maketitle

\section{Description}
ADAMS, the \textbf{A}dvanced \textbf{D}ata mining and \textbf{M}achine learning \textbf{S}ystem, is a scientifc workflow engine written in Java. Due to its modularization, it is easy to build custom applications that are targeted towards specific functionality, using the Maven build system. Currently available modules include support for Weka, MOA, R, image processing (ImageJ, JAI, ImageMagick, Gnuplot), PDF management and display, spreadsheet manipulation (CSV, Gnumeric, Excel, ODF), scripting (Groovy, Jython), GIS support (OpenStreetMap), Twitter, time-series analysis, network support (SSH, SCP, SFTP/FTP, Email), XML/HTML/JSON processing facilities, webservice capabilities.

In contrast to current versions of other workflow engines, like Kepler, KNIME or RapidMiner, where the user places operators on a canvas and connects them manually, ADAMS organizes the operators (or ``actors'' in ADAMS terminology) of a workflow automatically in a tree structure without explicit connections. Instead, so called ``control'' actors determine how data flows between actors. Examples of control actors are, e.g., Sequence, Branch, Tee, Trigger, IfThenElse and Switch. Apart from ``flow control'', there are two more aspects to an actor: ``functional'' (primitive actor or manages nested actors) and ``procedural'' in terms of input/output of data (standalone, source, transformer, sink).

Using a tree layout has advantages and disadvantages. As for the advantages, the layout is very compact, scales to thousands of actors, avoids having to manually rearrange the workflow in order to include additional actors (disconnect/reconnect), adding actors is context aware (data types of input and output limit what actors can be inserted), customizable rules for suggesting actors depending on context for common sequences of actors. Disadvantages are, the tree layout is less intuitive compared to a canvas-based approach, only supports 1-to-n connections. The 1-to-n limitation is mitigated using, callable actors (multiple actors can channel data into single actor using its name), containers for storing multiple outputs, variables, internal storage (re-using data in multiple locations). Variables can be either used in expressions, e.g., ones in evaluating mathematical formulas, or attached to parameters of operators. The latter allows for influencing the flow execution, e.g., for turning on/off of sub-flows. The scope of variables and internal storage can be limited using the LocalScope control actor.

Some further feature highlights: Though ADAMS is a data-driven workflow by design rather than an event-based one, i.e., actors get executed if there is data available for them to process, it is also possible to trigger sub-flows using cronjobs. This allows, for instance, for recurring clean-up operations. Interactive actors are very useful for developing workflow applications. These actors either prompt the user to enter or select a value, controlling sub-flow execution, or require the user to inspect data, e.g., visual inspection of images, influencing the data flow. By supporting scripting (Groovy/Jython), it is possible to quickly prototype new actors without the need of compiling Java code and restarting the workflow application. Once a workflow has been developed, it is not necessary to use the graphical user interface for executing it, the command-line can be used as well (e.g., for use in a headless server environment). Java-code generation from existing flows is possible as well.

\end{document}
